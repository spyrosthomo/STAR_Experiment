\chapter{Εισαγωγή}

Οι περισσότεροι από όσους δεν έχουν κάποια ιδιαίτερη εξοικείωση με την πειραμτική φυσική υψηλών ενεργειών, όπως κατά πάσα πιθανότητα και εμείς πριν τα μαθήματα του 7ου εξαμήνου, όταν ακούνε κάτι γι’ αυτήν, ο νους τους πάει στο CERN και το LHC και το πολύ μέχρι το Fermilab και το Desy. Φυσικά υπάρχει πληθώρα άλλων ενεργών και ανενεργών πειραμάτων τα οποία έχουν συνεισφέρει στην κατανόηση του εν λόγω αντικειμένου.

Η ιστορία ξεκινάει το 1910 από τον Rutherford, του οποίου οι φοιτητές στέλνοντας πυρήνες ηλίου 4MeV από ραδιενεργό $^{222}Ra$ σε φύλλα χρυσού ανακάλυψαν πως η μάζα του ατόμου περιέχεται σε μία μικρή του περιοχή, τον πυρήνα. Την ίδια χρονιά, ο Wilson σκέφτηκε τον Θάλαμο Αερίων, έναν πιο εξελιγένο ανιχνευτή από αυτόν του Rutherford, στον οποίο μειώνοντας απότομα την πίεση, οι ατμοί νερού συμπυκνώνονται τριγύρω από τις τροχιές ατόμων που έχουν ιονιστεί από την αλληλεπίδρασή τους με ενεργητικά σωματίδια. Την ίδια περίοδο με τα παραπάνω, το 1912, ο Victor Hess έβαλε μέσα σε ένα αερόστατο ένα ηλεκτρόμετρο Wolf, μία συσκευή για μέτρηση του ηλεκτρικού πεδίου άρα του ρυθμού ιονισμού των σωματιδίων αερίου σε ένα ερμητικά κλειστό δοχείο. Το αποτέλεσμα ήταν η ανίχνευση των ‘’Κοσμικών Ακτίνων’’, που στην πραγματικότητα πήραν το όνομά τους από τον Robert Millkan όταν το 1925 επιβεβαίωσε τον Hess. Έτσι, έως τα τέλη του 1940, η ενασχόληση με αυτές τις ακτίνες αποτέλεσε μία κύρια κατεύθυνση της φυσικής υψηλών ενεργειών, η οποία έγινε όλο και πιο λεπτομερής με την βελτίωση των ανιχνευτικών συσκευών.
 
Ωστόσο, υπήρχε η ανάγκη για περισσότερο έλεγχο στην δέσμη των σωματιδίων ώστε να μελετηθούν ακόμη πιο ενεργητικές δέσμες. Το 1931 στο Berkeley, ο E. Lawrence, κατασκεύασε το πρώτο Κύκλοτρο, έναν κυκλικό επιαχυντή ο οποίος χρησιμοποιεί μαγνητικό πεδίο για να στρίψει και μεταβαλλόμενο ηλεκτρικό πεδίο για να αυξήσει την ταχύτητα των σωματιδίων. Την επόμενη χρονία, πέτυχαν ενέργεια δέσμης 19MeV σε συσκευή ακτίνας 60 ιντσών. Μετά τον πόλεμο, ο Lawrence εκμεταλλευόμενος το κύρος των φυσικών στην Αμερική αύξησε τον προϋπολογισμό του εργαστηρίου του και χρησιμοποιώντας τους ισχυρούς μαγνήτες του Manhattan Project κατάφερε να αυξήσει την ενέργεια της δέσμης του σε 195MeV. Εδώ υπήρξε ένα εμπόδιο. Δεν μπορούσαν να φτιαχτούν μεγαλύτερα και άρα ισχυρότερα τέτοιου τύπου Κύκλοτρα που χρησιμοποιούσαν δύο επίπεδους μαγνήτες για την καμπύλωση της τροχιάς της δέσμης καθώς για δεδομένο μαγνητικό πεδίο, μεγαλύτερη ενέργεια σημαίνει μεγαλύτερη ακτίνα καμπυλότητας. Εφόσον ήταν αδύνατη η κατασκευή μεγαλύτερων επίπεδων μαγνητών που χρησιμοποιούνταν στα κύκλοτρα, έπρεπε να τοποθετούνται τοροειδείς μαγνήτες  κατά μήκος της τροχιάς προκειμένου η στροφή να γίνεται σε διακρτιτά σημεία και όχι καθ’ όλη την διάρκεια της κίνησης. 


Το 1947 εγκρίθηκε στις Η.Π.Α. η κατασκευή δύο Σύγχροτρων. Το ένα ήταν το Bevatron στο Berkeley (6.2GeV το 1954) και το άλλο λεγόταν Cosmotron, κατσκευάστηκε στο Brookhaven National Laboratory στο Long Island (3GeV το 1952). Παρεπιπτόντως, σε αυτό το Ερευνητικό Κέντρο υπάρχει σήμερα και το πείραμα STAR που ανιχνεύει γεγονότα από τον επιταχυντή RIHC. Στην Ρωσία υπήρχε από το 1957 το Synchrophasotron στην Ντουμπνά, που ήταν και ο μεγαλύτερος επιταχυντής της εποχής φτάνοντας δέσμες πρωτονίων σε ενέργειες των 10GeV. Στην υπόλοιπη Ευρώπη, ιδρύθηκε το CERN το 1952 και μέχρι το 1959 κατασκευάστηκε εκεί το PS (Proton Synchrotron) με ενέργεια στα 26GeV.
 	Την δεκαετία του 1970 και υπό το πείσμα για τεχνολογική υπεροχή έναντι της Ρωσίας, κατασκευάστηκε στην Αμερική το FermiLab, ένα Σύγχροτρο διαμέτρου 2 χιλιομέτρων που πέτυχε ενέργειες 500GeV μέχρι το 1976.
 	
	Ολοι οι παραπάνω επιταχυντές, καθώς και πολλοί άλλοι που δεν έχουν αναφερθεί, ήταν επιταχυντές πρωτονίων.
	 Ωστόσο, υπήρχαν και Σύγχροτρα ηλεκτρονίων τα οποία ήταν καταδικασμένα να λειτουργούν σε χαμηλότερες ενέργειες εξαιτίας της εκπομπής ακτινοβολίας synchrotron, η οποία εκπέμπεται κάθετα στην ταχύτητα σχετικιστικά επιταχυνόμενων σωματιδίων.
	  %από σχετικιστικά σωματίδια που επιταχύνονται κάθετα στην ταχύτητά τους 
	Ένα ακόμη μειονέκτημα ήταν ότι τα ηλεκτρόνια δεν αλληλεπιδρούν ισχυρά, άρα δεν μπορούν να χρησιμοποιηθούν για την απευθείας μελέτη της ισχυρής πυρηνικής αλληλεπίδρασης.	Για την υπέρβαση του προβλήματος της ακτινοβλίας Σύγχροτρου, κατασκευάστηκε κοντα στο Stanford ο γραμμικός επιταχυντής SLAC.

Οι παραπάνω συσκευές ήταν, όπως αναφέρεται, επιταχυντές, δηλαδή επιτάχυναν μία δέσμη σωματιδίων κατευθύνοντάς την σε έναν ακίνητο στόχο. Όμως με αυτόν τον τρόπο η διαθέσιμη ενέργεια στο σύστημα του κέντρου μάζας είναι ανάλογη με την τετραγωνική ρίζα της ενέργειας της δέσμης, π.χ. αν η ενέργεια μίας δέσμης ηλεκτρονίων είναι 500GeV, η ενέργεια στο σύστημα κέντρου μάζας είναι περίπου 20GeV. Αυτό σημαίνει πως κατά την κρούση δεν μπορούν να παραχθούν σωματίδια με μάζα μεγαλύτερη των 20GeV. Αυτό το πρόβλημα μπορεί να ξεπεραστεί στους Colliders. Εκεί έχουμε δύο δέσμες οι οποίες επιταχύνονται η μία αντίθετα από την άλλη αυξάνοντας έτσι την διαθέσιμη ενέργεια στο κέντρο μάζας άρα και την μάζα των δυνητικά παραγόμενων σωματιδίων. 

	Παραμένωντας στις εξελίξεις στις Η.Π.Α., την δεκαετία του 80’ έπειτα από μία σχετικά μικρή αποτυχία στην κατασκευή ενός επιταχυντή, έρχεται μία νέα και πιο επιβλητική αποτυχία, η Α-2. Καθώς η πρώτη αποτυχία ( η εξέλιξη της οποίας θα μας απασχολήσει εκτενώς και σχεδόν αποκλειστικά στην συνέχεια της εργασίας ) ολοκλρώθηκε, το 1987 πάρθηκε η απόφαση να σταματήσουν οι αναβαθμίσεις που γίνονταν εκείνη την εποχή στο Tevatron του Fermilab (τότε 1.8TeV) προκειμένου να διατεθούν οι πόροι σε κάτι αναπάντεχα φιλόδοξο. Αυτό ήταν ο Superconducnting Super Collider (SSC) που θα κατασκευαζοταν στο Texas και θα είχε περίμετρο 87 χιλιόμετρα με συνολική ενέργεια στο σύστημα κέντρου μάζας 20TeV. Ξεκίνησε η κατασκευή το 1991 με εκτιμώμενο κόστος $\$4.4$ δισεκατομμύρια. 
	Στην συνέχεια λόγω απαιτούμενων αλλαγών ανέβηκε στα \$8.25 και το 1993 είχε φτάσει στα $\$$11 δισεκατομμύρια. Έτσι για πολιτικούς, αλλά και επιστημονικούς λόγους, φοβούμενοι ότι θα υποβαθμιστούν άλλοι επιστημονικοί τομείς λόγω ελλειπούς χρηματοδότησης, οι εργασίες σταμάτησαν ενώ είχαν ήδη ξοδευτεί $\$2$ δυσεκατομμύρια και ενώ είχε γίνει η εκσκαφή 22 χιλιομέτρων σήραγγας. 
	Η τραγική αυτή κατάληξη  οδήγησε τον SSC να έχει πρωταγωνιστικό ρόλο ακόμη και σε βιβλία και τραγούδια (‘Supercollider - Tribe’, ‘A Hole in Texas – Herman Wouk’, ‘Einstein’s Bridge – John G. Craner’).
	
	Τώρα σχετικά με την πρώτη αποτυχία. Το 1978 στο Brookhaven National Laboratory (BNL) στο Long Island ξεκίνησαν οι εκσκαφές για ένα τούνελ 4 χιλιομέτρων όπου θα κατασκευαζόταν ένας 200GeV-200GeV Collider πρωτονίου-πρωτονίου χρησιμοποιώντας για πρώτη φορά υπεραγώγιμους μαγνήτες, ο ISABELLE. 
	Παρ’ όλο που η πρώτη δοκιμή των μαγνητών ήταν επιτυχής, στην συνέχεια προέκυψαν κάποια τεχνικά προβλήματα και η κατασκευή του επιταχυντή σταμάτησε αφού το ανταγωνιστικό πρόγραμμα στην Ευρώπη, SPS λειτουργούσε από το 1983 και επίσης υπήρχε και το εξαιρετικά κοστοβόρο πρόγραμμα του SSP. 
	Όμως, το τούνελ των τεσσάρων χιλιομέτρων είχε παραμείνει. Έτσι, έγιναν προτάσεις για την κατασκευή ενός νέου Collider, του Relativistic Heavy Ion Collider (RHIC) και το 1991 άρχισαν οι κατασκευές που ολοκληρώθηκαν το 1999 με συνολικό κόστος \$617 εκατομμύρια. 
Ο RHIC ήταν ο πρώτος Collider στον οποίον συγκρούονταν βαρέα ιόντα, κυρίως χρυσού τα οποία υπό κατάλληλες συνθήκες παράγουν το λεγόμενο Πλάσμα Κουάρκ-Γλουονίων, το οποίο καλούμε πλάσμα διότι αποτελείται από ελεύθερα Κουάρκ και Γλουόνια όπως το Ηλεκτρομαγνητικό πλάσμα αποτελείται από ιόντα και ηλεκτρόνια. 
Έτσι, ένας βασικός στόχος είναι η μελέτη αυτού του πλάσματος το οποίο υποθέτουμε πως υπήρχε τα πρώτα κλάσματα του δευτερολέπτου μετά το Big Bang καθώς απαιτούνται συνθήκες πολύ υψηλής θερμοκρασίας για την δημιουργία του. Επίσης, ο RHIC μπορεί να συγκρούει πολωμένα πρωτόνια με στόχο την εξερεύνηση της προέλευσης του σπιν του πρωτονίου.  Πρωτού όμως φτάσουμε στον τρόπο με τον οποίο γίνεται η μελέτη και ανίχνευση του πλάσματος και των πρωτονίων, θα πρέπει να δούμε συνοπτικά πως δουλεύει ο RHIC.
