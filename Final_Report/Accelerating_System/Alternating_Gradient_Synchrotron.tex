\subsection{Alternating Gradient Synchrotron (AGS)}


	Το AGS κατασκευάστηκετο 1960 και για τα επόμενα οκτώ χρόνια ήταν ο επιταχυντής με την υψηλότερη ενέργεια, επιταχύνοντας αρχικά πρωτόνια στα 33GeV και έπειτα ιόντα με μία ενδεικτή ενέργεια για τον χρυσό στα 14.5GeV/n.
	Το συνολικό μήκος της περιφέρειάς του είναι 842.90m και αποτελείται από 240 μαγνήτες. Οι δέσμες εισάγονται στον AGS σε 24 bunches τα οποιά γίνονται μία δέσμη και εν τέλει ξαναχωρίζονται σε 4 για να οδηγηθούν στον RHIC.
	
	 Είναι ο πρώτος επιταχυντής που χρησιμοποίησε την αρχή της Ισχυρής Εστίασης (Strong Focusing/Alternating Gradient Focusing Principle) την οποία σκέφτηκε για πρώτη φορά ο Ν. Χριστόφιλου το 1949 και έπειτα, το 1952 ανακαλύφθηκε ανεξάρτητα από τρεις επιστήμονες του BNL. 
	 Η εν λόγω αρχή είναι αυτή που χρησιμοποιείται και στον Booster Synchrotron και σύμφωνα με αυτή, αν τοποθετήσουμε διαδοχικά τετραπολικούς μαγνήτες αντίθετης πολικότητας, εκ των οποίον ο ένας εστιάζει και ο άλλος διευρύνει την δέσμη, το τελικό αποτέλεσμα θα είναι η εστίασή της και ταυτόχρονα η μείωση του μεγέθος των μαγνητών που ελαττώνει σημαντικά το κόστος κατασκευής. 
	
	Το κάθε κελί του AGS είναι της μορφής FOFDOD και πέρα από αυτά, αποτελείται επίσης από 24 ευθύγραμμα τμήματα μήκους 3.15m. Το μαγνητικό πεδίο που καμπυλώνει την δέσμη πρέπει να είναι μικρότερο όταν αυτή εισάγεται στον επιταχυντή έχοντας μικρή ενέργεια και μεγαλύτερο όταν εξάγεται προς τον RHIC. Αυτό μπορεί να εξηγηθεί πρόχειρα από τον 2ο νόμο του Newton από τον οποίο προκύπτει για την ακτίνα ότι $1/R = ecB_\perp/cp$. Αυτό σημαίνει πως για να διατηρηθεί η ακτίνα σταθερή θα πρέπει καθώς αυξάνεται η ενέργεια άρα και η ορμή του σωματιδίου, να αυξάνεται και το μαγνητικό πεδίο που στρίβει το σωματίδιο.
	
	Τα πρώτα σύγχροτρα ήταν ακριβώς κυκλικά μέχρι που κατασκευάστηκαν τα Bevatron και Cosmotron τα οποία είχαν και ευθύγραμμα τμήματα μεταξύ των μαγνητών, διότι εν γένει είναι ευκολότερη η συγκράτηση, η επιτάχυνση και η διατήρηση των χαρακτηρηστικών της δέσμης όταν υπάρχουν ευθύγραμμα τμήματα επιτάχυνσης.
	Ωστόσο, αν αυξηθούν αρκετά τα εν λόγω τμήματα δημιουργούνται ορισμένα προβλήματα με το σημαντικότερο να είναι πως θεωρητικά, την εποχή κατασκευής του AGS περίμεναν μία ασταθεια στην δέσμη για μία συγκεκριμένη τιμή της ενέργειας (transition energy) η οποία βέβαια εξαφανίζεται για μικρότερες και μεγαλύτερες τιμές ενεργειών. 
	Αυτή η αστάθεια προβλημάτιζε τους επιστήμονες που σχεδίαζαν το AGS και παρ'όλο που θεωρητικά προβλεπόταν πως η μετάβαση από αυτή ήταν πολύ γρήγορη, τους οδήγησε στο να την ελέγξουν με ένα μικρότερο σύγχροτρο ηλεκτρονίων της τάξης MeV. Αφού επιβεβαίωσαν την θεωρητική πρόβλεψη συνέχισαν με την κατασκευή του AGS. Στα επόμενα 30 χρόνια λειτουργίας του προέκυψαν από το AGS τρία βραβεία Νόμπελ, σχετιζόμενα με την παραβίαση της CP συμμετρίας, την ανακάλυψη του μιονικού νετρίου και του σωματιδίου J/$\psi$.